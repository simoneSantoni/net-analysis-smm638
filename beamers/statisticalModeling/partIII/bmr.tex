\documentclass[notes, aspectratio=1610]{beamer}

% =================================== Colors ================================
\definecolor{base_c}{rgb}{
	0.1411764705882353, 0.6235294117647059, 0.33725490196078434
	}
\definecolor{comp_c}{rgb}{
	0.6235294117647059, 0.1411764705882353, 0.42745098039215684
	}
\definecolor{tri_1}{rgb}{
	0.33725490196078434, 0.1411764705882353, 0.6235294117647059
	}
\definecolor{tri_2}{rgb}{
	0.6235294117647059, 0.33725490196078434, 0.1411764705882353
	}
\definecolor{white}{rgb}{1,1,1}
\usepackage{color, colortbl}

% ================================ Text boxes ==============================
\usepackage[most]{tcolorbox}

% ================================ Symbols ==============================
%\usepackage{bbding}
%\usepackage{pifont}
%\usepackage{wasysym}
%\usepackage{amssymb}

% ========================= Theme =========================================
\usetheme{Berkeley}
\usecolortheme{spruce}
% \setbeamercolor{itemize item}{fg=comp_c,bg=white}
% \setbeamercolor{enumerate item}{fg=comp_c,bg=white}

% ========================= Essential packages ============================
% \usepackage{hyperref}
% \hypersetup{
%     colorlinks = false,
%     linkcolor = comp_c,
%     citecolor = tri_1,
%     filecolor = tri_2,
%     urlcolor = yellow
% }

% ========================= Frame notes systm ============================
%\usepackage{pgfpages}
%\setbeameroption{show notes on second screen}

% ========================= Plotting ======================================
\usepackage{calc}
\usepackage{tikz}
\usetikzlibrary{arrows,
                arrows.meta,
                calc,
		chains,
                quotes,
                positioning,
		shapes,
		topaths,
                shapes.geometric}
\usepackage{graphicx}
\usepackage{graphics}
\usepackage{pgfplots}
\pgfplotsset{width=7cm,compat=1.17}
\usepackage{venndiagram}
% tikz exhibits
\usepackage{tikz}
\usetikzlibrary{shapes,arrows,fit,calc,positioning}
\tikzstyle{box} = [
	draw,
	rectangle,
	rounded corners,
	thick,
	node distance=7em,
	text width=6em,
	text centered,
	minimum height=3.5em
	]
\tikzstyle{container} = [draw, rectangle, dashed, inner sep=2em]
\tikzstyle{lineSq} = [draw, thick, -latex']
\tikzset{
	%Define standard arrow tip
	>=stealth',
	%Define style for boxes
	punkt/.style={
	       rectangle,
	       rounded corners,
	       draw=black, thick,
	       text width=8em,
	       minimum height=3em,
	       text centered},
	punkbig/.style={
		   rectangle,
		   rounded corners,
		   draw=black, thick,
		   text width=5em,
		   minimum height=5em,
		   text centered},
	punkb/.style={
		   rectangle,
		   rounded corners,
		   %draw=black, thick,
		   text width=8em,
		   minimum height=3em,
		   text centered},
	punkc/.style={
	       circle,
	       rounded corners,
	       draw=black, thick,
	       text width=1.5em,
	       minimum height=1.5em,
	       text centered},
	% Define arrow style
	pil/.style={
	       ->,
	       thick,
	       shorten <=2pt,
	       shorten >=2pt,}
}

% ============================= Network viz ===============================
\usepackage{tikz-network}

%% ============================== Tabular =================================
\usepackage{booktabs}
\usepackage{tabularx,ragged2e}
\usepackage{array}
\usepackage{multirow}
\usepackage{siunitx}
  \sisetup{detect-all}
\usepackage{adjustbox}
\usepackage{rotating}
\usepackage{threeparttable}
\usepackage[justification=centering]{caption}

%% ========================== Coding snippets =============================
% Default fixed font does not support bold face
%\usepackage{minted}
%\usemintedstyle{vs}

%% ============================= Math =====================================
\usepackage{mathtools}

% ========================= Infor on authors ==============================
\title[Network Mechanisms]
{Network Mechanisms and Network Structures}
\subtitle{The Principle of Homophily}
\author{S.~Santoni\inst{1}\inst{2}}
\institute{
	\inst{1}%
	Bayes Business School
	\and
	\inst{2}%
	Soundcloud
	}
\date{MSc in Business Analytics, 2022/23}

% ========================= TOC  ==========================================
\AtBeginSection[]
{
	\begin{frame}
		       \frametitle{Outline}
		       \tableofcontents[currentsection,currentsubsection]
	\end{frame}
}

% ========================= References ===================================
\usepackage[style=numeric,backend=biber]{biblatex}
\addbibresource{bibliography.bib}

% ========================= Document  ====================================
\begin{document}

\begin{frame}
	\titlepage
\end{frame}

\begin{frame}{Outline}
	\tableofcontents
\end{frame}

% =========================== Session 7 wrapup =============================
\section{Wrap Up on Network Change}

\begin{frame}
	\frametitle{Network Theories across the Various Weeks of SMM638}
	\begin{table}
		\begin{tabular}[c]{l|c|c|c|c|c|c|c|c}
			\textbf{Network theory} & 
			\textbf{2} & 
			\textbf{3} & 
			\textbf{4} & 
			\textbf{5} & 
			\textbf{6} & 
			\textbf{7} & 
			\textbf{9} & 
			\textbf{10}\\
			\hline
			Value creation &  & $\bullet$ & $\bullet$ &  &  &  &  & \\	
			Coordination &  &  &  & $\bullet$ &  &  &  & \\	
			Network change &  &  &  &  & $\bullet$ & $\bullet$ & $\bullet$ & $\bullet$\\	
			Contagion &  &  &  &  &  & $\bullet$ &  & $\bullet$ \\	
		\end{tabular}
	\end{table}
\end{frame}

\begin{frame}
	\frametitle{Groups of Network Theories}
	\centering
	\small
	\begin{table}
		\begin{center}
			\begin{tabular}[c]{l|l|l}
				% \hline
				\textbf{Underlying model} & 
				\textbf{Social capital} &
				\textbf{Social homogeneity}\\
				\hline
				\textbf{Network flow} &
				Capitalization (value creation) & 
				\cellcolor{blue!20}{\color{black}{Contagion}}  \\
				\hline
				\textbf{Network architecture} & 
				Coordination & 
				Adaptation  (network change)\\
				% \hline
			\end{tabular}
		\end{center}
	\end{table}

	\vspace{1em}

	\raggedright \small Source is~\cite[][page 47]{scott2011}
\end{frame}

\begin{frame}{Why do Network Mechanisms Matter?}{}
	\begin{enumerate}
		\item Network mechanisms can explain why networks look how 
		they look!
		\item If we know the network mechanisms, we can predict the
		evolution of a network. E.g.,
		\begin{itemize}
			\item 
			Who will pay attention to which market offers
			\item 
			Who will date whom
			\item 
			Who will adopt or reject whom's opinion
		\end{itemize}
	\end{enumerate}
\end{frame}

\begin{frame}
	\frametitle{What Are We Trying to Achieve in Sessions 6 and 7?}
	\begin{table}
		\begin{tabular}[c]{l|c|c|c|c|c|c|c|c}
			\textbf{Network theory} & 
			\textbf{2} & 
			\textbf{3} & 
			\textbf{4} & 
			\textbf{5} & 
			\textbf{6} & 
			\textbf{7} & 
			\textbf{9} & 
			\textbf{10}\\
			\hline
			Value creation &  & $\bullet$ & $\bullet$ &  &  &  &  & \\	
			Coordination &  &  &  & $\bullet$ &  &  &  & \\	
			Network change &  &  &  &  & $\bullet$ & $\bullet$ & $\bullet$ & $\bullet$\\	
			Contagion &  &  &  &  &  & $\bullet$ &  & $\bullet$ \\	
		\end{tabular}
	\end{table}

	\pause

	\begin{itemize}
		\item In session 6, we tried to discover the 
		mixture of network 
		mechanisms accounting for Soundcloud's network structure
		\pause
		\item In session 7, we learned how to quantify a key 
		network mechanism and will connect it to the problem 
		of contagion (critically important to explain offering 
		popularity in markets)
	\end{itemize}
\end{frame}

\subsection{Session 6}

\begin{frame}
	\frametitle{A Real-World Example: The Soundcloud Networks}
	\begin{columns}
		\begin{column}{0.55\textwidth}
			\begin{figure}[!htbp]
				\begin{tikzpicture}[scale=0.9, transform shape, multilayer=3d]
				  %% separation between Layers
				  \SetLayerDistance{-4.5}
				  %% layer for activities
				  \begin{Layer}[layer=1]
				    %% plane
				    \Plane[x = 0, y = 0, width = 3, height = 6, layer = 1,
				      color = white, opacity = 0, style = {thick, gray}];
				    %% label
				    \node at (0, 3) [above right] {Offerings};
				  \end{Layer}
				  %% layer for actors
				  \begin{Layer}[layer=2]
				    \Plane[x = 0, y = 0, width = 3, height = 6, layer = 2,
				      color = white, opacity = 0, style = {thick, gray}];
				    %% plane
				    \node at (0, 0) [below right] {Followership network};
				  \end{Layer}
				  %% vertices
				  \Vertices{data/soundcloud_vertices.csv}
				  %% edges
				  \Edges{data/soundcloud_edges.csv}
				\end{tikzpicture}
			      \end{figure}
		\end{column}
		\begin{column}{0.4\textwidth}
			\begin{small}

			Some key general points emerging from the analysis of the 
			Soundcloud example:
			\begin{itemize}
				\item
				The same pair of nodes can be connected 
				because of multiple relationships (i.e., `like,'
				`repost,' `comment')
				\item 
				The nodes of a network may have the 
				same type (e.g., `following') or 
				different types (e.g., `like')
				\item Analytically separated networks may be 
				correlated (e.g., one tends to like her/his 
				followings' likes)
			\end{itemize}

			\end{small}
		\end{column}
	\end{columns}
\end{frame}

\begin{frame}{Network Mechanisms in the Followership Network}{Popularity effects}
	\begin{columns}[t]
		\begin{column}{0.5\textwidth}
			\centering 
			\textbf{Indegree popularity}
			
			\input{images/indegree_pop.pgf}

			\raggedright
			\small
			\textit{Synopsis:} certain nodes may develop a cumulative 
			advantage --- a so-called rich-get-richer effect
		\end{column}
		\begin{column}{0.5\textwidth}
			\centering 
			\textbf{Outdegree popularity}
			
			\input{images/outdegree_pop.pgf}

			\raggedright
			\small
			\textit{Synopsis:} certain nodes may develop a tendency 
			to connect with many others
		\end{column}
	\end{columns}
\end{frame}

\begin{frame}{Network Mechanisms in the Followership Network}{Reciprocity effect}
	\centering 
	\input{images/reciprocity.pgf}

	\raggedright
	\small 
	\textit{Synopsis:} pairs of nodes may show a tendency to follow each 
	other --- that is, to reciprocate each other's following choice 
\end{frame}

\begin{frame}{Network Mechanisms in the Followership Network}{Triadic closure}
	\centering 
	\input{images/triadic_closure.pgf}

	\small
	\textit{Synopsis:} ``My following's following is my following'' 
\end{frame}

\begin{frame}{Network Mechanisms in the Followership Network}{Balance}
	\centering 
	%% Creator: Matplotlib, PGF backend
%%
%% To include the figure in your LaTeX document, write
%%   \input{<filename>.pgf}
%%
%% Make sure the required packages are loaded in your preamble
%%   \usepackage{pgf}
%%
%% Also ensure that all the required font packages are loaded; for instance,
%% the lmodern package is sometimes necessary when using math font.
%%   \usepackage{lmodern}
%%
%% Figures using additional raster images can only be included by \input if
%% they are in the same directory as the main LaTeX file. For loading figures
%% from other directories you can use the `import` package
%%   \usepackage{import}
%%
%% and then include the figures with
%%   \import{<path to file>}{<filename>.pgf}
%%
%% Matplotlib used the following preamble
%%   \usepackage{fontspec}
%%   \setmainfont{DejaVuSerif.ttf}[Path=\detokenize{/Users/sbbk475/opt/anaconda3/lib/python3.9/site-packages/matplotlib/mpl-data/fonts/ttf/}]
%%   \setsansfont{DejaVuSans.ttf}[Path=\detokenize{/Users/sbbk475/opt/anaconda3/lib/python3.9/site-packages/matplotlib/mpl-data/fonts/ttf/}]
%%   \setmonofont{DejaVuSansMono.ttf}[Path=\detokenize{/Users/sbbk475/opt/anaconda3/lib/python3.9/site-packages/matplotlib/mpl-data/fonts/ttf/}]
%%
\begingroup%
\makeatletter%
\begin{pgfpicture}%
\pgfpathrectangle{\pgfpointorigin}{\pgfqpoint{1.750000in}{1.710000in}}%
\pgfusepath{use as bounding box, clip}%
\begin{pgfscope}%
\pgfsetbuttcap%
\pgfsetmiterjoin%
\definecolor{currentfill}{rgb}{1.000000,1.000000,1.000000}%
\pgfsetfillcolor{currentfill}%
\pgfsetlinewidth{0.000000pt}%
\definecolor{currentstroke}{rgb}{1.000000,1.000000,1.000000}%
\pgfsetstrokecolor{currentstroke}%
\pgfsetstrokeopacity{0.000000}%
\pgfsetdash{}{0pt}%
\pgfpathmoveto{\pgfqpoint{0.000000in}{0.000000in}}%
\pgfpathlineto{\pgfqpoint{1.750000in}{0.000000in}}%
\pgfpathlineto{\pgfqpoint{1.750000in}{1.710000in}}%
\pgfpathlineto{\pgfqpoint{0.000000in}{1.710000in}}%
\pgfpathlineto{\pgfqpoint{0.000000in}{0.000000in}}%
\pgfpathclose%
\pgfusepath{fill}%
\end{pgfscope}%
\begin{pgfscope}%
\pgfpathrectangle{\pgfqpoint{0.100000in}{0.100000in}}{\pgfqpoint{1.550000in}{1.510000in}}%
\pgfusepath{clip}%
\pgfsetroundcap%
\pgfsetroundjoin%
\pgfsetlinewidth{1.003750pt}%
\definecolor{currentstroke}{rgb}{0.000000,0.000000,0.000000}%
\pgfsetstrokecolor{currentstroke}%
\pgfsetdash{}{0pt}%
\pgfpathmoveto{\pgfqpoint{0.320625in}{0.314932in}}%
\pgfpathquadraticcurveto{\pgfqpoint{0.554722in}{0.542987in}}{\pgfqpoint{0.777696in}{0.760207in}}%
\pgfusepath{stroke}%
\end{pgfscope}%
\begin{pgfscope}%
\pgfpathrectangle{\pgfqpoint{0.100000in}{0.100000in}}{\pgfqpoint{1.550000in}{1.510000in}}%
\pgfusepath{clip}%
\pgfsetroundcap%
\pgfsetroundjoin%
\definecolor{currentfill}{rgb}{0.000000,0.000000,0.000000}%
\pgfsetfillcolor{currentfill}%
\pgfsetlinewidth{1.003750pt}%
\definecolor{currentstroke}{rgb}{0.000000,0.000000,0.000000}%
\pgfsetstrokecolor{currentstroke}%
\pgfsetdash{}{0pt}%
\pgfpathmoveto{\pgfqpoint{0.718519in}{0.741337in}}%
\pgfpathlineto{\pgfqpoint{0.777696in}{0.760207in}}%
\pgfpathlineto{\pgfqpoint{0.757286in}{0.701543in}}%
\pgfpathlineto{\pgfqpoint{0.718519in}{0.741337in}}%
\pgfpathlineto{\pgfqpoint{0.718519in}{0.741337in}}%
\pgfpathclose%
\pgfusepath{stroke,fill}%
\end{pgfscope}%
\begin{pgfscope}%
\pgfpathrectangle{\pgfqpoint{0.100000in}{0.100000in}}{\pgfqpoint{1.550000in}{1.510000in}}%
\pgfusepath{clip}%
\pgfsetroundcap%
\pgfsetroundjoin%
\pgfsetlinewidth{1.003750pt}%
\definecolor{currentstroke}{rgb}{0.000000,0.000000,0.000000}%
\pgfsetstrokecolor{currentstroke}%
\pgfsetdash{}{0pt}%
\pgfpathmoveto{\pgfqpoint{0.289410in}{0.338011in}}%
\pgfpathquadraticcurveto{\pgfqpoint{0.554746in}{0.854989in}}{\pgfqpoint{0.812993in}{1.358153in}}%
\pgfusepath{stroke}%
\end{pgfscope}%
\begin{pgfscope}%
\pgfpathrectangle{\pgfqpoint{0.100000in}{0.100000in}}{\pgfqpoint{1.550000in}{1.510000in}}%
\pgfusepath{clip}%
\pgfsetroundcap%
\pgfsetroundjoin%
\definecolor{currentfill}{rgb}{0.000000,0.000000,0.000000}%
\pgfsetfillcolor{currentfill}%
\pgfsetlinewidth{1.003750pt}%
\definecolor{currentstroke}{rgb}{0.000000,0.000000,0.000000}%
\pgfsetstrokecolor{currentstroke}%
\pgfsetdash{}{0pt}%
\pgfpathmoveto{\pgfqpoint{0.762912in}{1.321411in}}%
\pgfpathlineto{\pgfqpoint{0.812993in}{1.358153in}}%
\pgfpathlineto{\pgfqpoint{0.812338in}{1.296043in}}%
\pgfpathlineto{\pgfqpoint{0.762912in}{1.321411in}}%
\pgfpathlineto{\pgfqpoint{0.762912in}{1.321411in}}%
\pgfpathclose%
\pgfusepath{stroke,fill}%
\end{pgfscope}%
\begin{pgfscope}%
\pgfpathrectangle{\pgfqpoint{0.100000in}{0.100000in}}{\pgfqpoint{1.550000in}{1.510000in}}%
\pgfusepath{clip}%
\pgfsetroundcap%
\pgfsetroundjoin%
\pgfsetlinewidth{1.003750pt}%
\definecolor{currentstroke}{rgb}{0.000000,0.000000,0.000000}%
\pgfsetstrokecolor{currentstroke}%
\pgfsetdash{}{0pt}%
\pgfpathmoveto{\pgfqpoint{1.429375in}{0.314932in}}%
\pgfpathquadraticcurveto{\pgfqpoint{1.195278in}{0.542987in}}{\pgfqpoint{0.972304in}{0.760207in}}%
\pgfusepath{stroke}%
\end{pgfscope}%
\begin{pgfscope}%
\pgfpathrectangle{\pgfqpoint{0.100000in}{0.100000in}}{\pgfqpoint{1.550000in}{1.510000in}}%
\pgfusepath{clip}%
\pgfsetroundcap%
\pgfsetroundjoin%
\definecolor{currentfill}{rgb}{0.000000,0.000000,0.000000}%
\pgfsetfillcolor{currentfill}%
\pgfsetlinewidth{1.003750pt}%
\definecolor{currentstroke}{rgb}{0.000000,0.000000,0.000000}%
\pgfsetstrokecolor{currentstroke}%
\pgfsetdash{}{0pt}%
\pgfpathmoveto{\pgfqpoint{0.992714in}{0.701543in}}%
\pgfpathlineto{\pgfqpoint{0.972304in}{0.760207in}}%
\pgfpathlineto{\pgfqpoint{1.031481in}{0.741337in}}%
\pgfpathlineto{\pgfqpoint{0.992714in}{0.701543in}}%
\pgfpathlineto{\pgfqpoint{0.992714in}{0.701543in}}%
\pgfpathclose%
\pgfusepath{stroke,fill}%
\end{pgfscope}%
\begin{pgfscope}%
\pgfpathrectangle{\pgfqpoint{0.100000in}{0.100000in}}{\pgfqpoint{1.550000in}{1.510000in}}%
\pgfusepath{clip}%
\pgfsetroundcap%
\pgfsetroundjoin%
\pgfsetlinewidth{1.003750pt}%
\definecolor{currentstroke}{rgb}{0.000000,0.000000,0.000000}%
\pgfsetstrokecolor{currentstroke}%
\pgfsetdash{}{0pt}%
\pgfpathmoveto{\pgfqpoint{1.460590in}{0.338011in}}%
\pgfpathquadraticcurveto{\pgfqpoint{1.195254in}{0.854989in}}{\pgfqpoint{0.937007in}{1.358153in}}%
\pgfusepath{stroke}%
\end{pgfscope}%
\begin{pgfscope}%
\pgfpathrectangle{\pgfqpoint{0.100000in}{0.100000in}}{\pgfqpoint{1.550000in}{1.510000in}}%
\pgfusepath{clip}%
\pgfsetroundcap%
\pgfsetroundjoin%
\definecolor{currentfill}{rgb}{0.000000,0.000000,0.000000}%
\pgfsetfillcolor{currentfill}%
\pgfsetlinewidth{1.003750pt}%
\definecolor{currentstroke}{rgb}{0.000000,0.000000,0.000000}%
\pgfsetstrokecolor{currentstroke}%
\pgfsetdash{}{0pt}%
\pgfpathmoveto{\pgfqpoint{0.937662in}{1.296043in}}%
\pgfpathlineto{\pgfqpoint{0.937007in}{1.358153in}}%
\pgfpathlineto{\pgfqpoint{0.987088in}{1.321411in}}%
\pgfpathlineto{\pgfqpoint{0.937662in}{1.296043in}}%
\pgfpathlineto{\pgfqpoint{0.937662in}{1.296043in}}%
\pgfpathclose%
\pgfusepath{stroke,fill}%
\end{pgfscope}%
\begin{pgfscope}%
\pgfpathrectangle{\pgfqpoint{0.100000in}{0.100000in}}{\pgfqpoint{1.550000in}{1.510000in}}%
\pgfusepath{clip}%
\pgfsetbuttcap%
\pgfsetroundjoin%
\definecolor{currentfill}{rgb}{1.000000,1.000000,1.000000}%
\pgfsetfillcolor{currentfill}%
\pgfsetlinewidth{1.003750pt}%
\definecolor{currentstroke}{rgb}{1.000000,1.000000,1.000000}%
\pgfsetstrokecolor{currentstroke}%
\pgfsetdash{}{0pt}%
\pgfsys@defobject{currentmarker}{\pgfqpoint{-0.120281in}{-0.120281in}}{\pgfqpoint{0.120281in}{0.120281in}}{%
\pgfpathmoveto{\pgfqpoint{0.000000in}{-0.120281in}}%
\pgfpathcurveto{\pgfqpoint{0.031899in}{-0.120281in}}{\pgfqpoint{0.062496in}{-0.107608in}}{\pgfqpoint{0.085052in}{-0.085052in}}%
\pgfpathcurveto{\pgfqpoint{0.107608in}{-0.062496in}}{\pgfqpoint{0.120281in}{-0.031899in}}{\pgfqpoint{0.120281in}{0.000000in}}%
\pgfpathcurveto{\pgfqpoint{0.120281in}{0.031899in}}{\pgfqpoint{0.107608in}{0.062496in}}{\pgfqpoint{0.085052in}{0.085052in}}%
\pgfpathcurveto{\pgfqpoint{0.062496in}{0.107608in}}{\pgfqpoint{0.031899in}{0.120281in}}{\pgfqpoint{0.000000in}{0.120281in}}%
\pgfpathcurveto{\pgfqpoint{-0.031899in}{0.120281in}}{\pgfqpoint{-0.062496in}{0.107608in}}{\pgfqpoint{-0.085052in}{0.085052in}}%
\pgfpathcurveto{\pgfqpoint{-0.107608in}{0.062496in}}{\pgfqpoint{-0.120281in}{0.031899in}}{\pgfqpoint{-0.120281in}{0.000000in}}%
\pgfpathcurveto{\pgfqpoint{-0.120281in}{-0.031899in}}{\pgfqpoint{-0.107608in}{-0.062496in}}{\pgfqpoint{-0.085052in}{-0.085052in}}%
\pgfpathcurveto{\pgfqpoint{-0.062496in}{-0.107608in}}{\pgfqpoint{-0.031899in}{-0.120281in}}{\pgfqpoint{0.000000in}{-0.120281in}}%
\pgfpathlineto{\pgfqpoint{0.000000in}{-0.120281in}}%
\pgfpathclose%
\pgfusepath{stroke,fill}%
}%
\begin{pgfscope}%
\pgfsys@transformshift{0.234504in}{0.231033in}%
\pgfsys@useobject{currentmarker}{}%
\end{pgfscope}%
\begin{pgfscope}%
\pgfsys@transformshift{1.515496in}{0.231033in}%
\pgfsys@useobject{currentmarker}{}%
\end{pgfscope}%
\begin{pgfscope}%
\pgfsys@transformshift{0.875000in}{0.855000in}%
\pgfsys@useobject{currentmarker}{}%
\end{pgfscope}%
\begin{pgfscope}%
\pgfsys@transformshift{0.875000in}{1.478967in}%
\pgfsys@useobject{currentmarker}{}%
\end{pgfscope}%
\end{pgfscope}%
\begin{pgfscope}%
\definecolor{textcolor}{rgb}{0.000000,0.000000,0.000000}%
\pgfsetstrokecolor{textcolor}%
\pgfsetfillcolor{textcolor}%
\pgftext[x=0.234504in,y=0.231033in,,]{\color{textcolor}\sffamily\fontsize{12.000000}{14.400000}\selectfont 0}%
\end{pgfscope}%
\begin{pgfscope}%
\definecolor{textcolor}{rgb}{0.000000,0.000000,0.000000}%
\pgfsetstrokecolor{textcolor}%
\pgfsetfillcolor{textcolor}%
\pgftext[x=1.515496in,y=0.231033in,,]{\color{textcolor}\sffamily\fontsize{12.000000}{14.400000}\selectfont 1}%
\end{pgfscope}%
\begin{pgfscope}%
\definecolor{textcolor}{rgb}{0.000000,0.000000,0.000000}%
\pgfsetstrokecolor{textcolor}%
\pgfsetfillcolor{textcolor}%
\pgftext[x=0.875000in,y=0.855000in,,]{\color{textcolor}\sffamily\fontsize{12.000000}{14.400000}\selectfont 2}%
\end{pgfscope}%
\begin{pgfscope}%
\definecolor{textcolor}{rgb}{0.000000,0.000000,0.000000}%
\pgfsetstrokecolor{textcolor}%
\pgfsetfillcolor{textcolor}%
\pgftext[x=0.875000in,y=1.478967in,,]{\color{textcolor}\sffamily\fontsize{12.000000}{14.400000}\selectfont 3}%
\end{pgfscope}%
\end{pgfpicture}%
\makeatother%
\endgroup%


	\small 
	\textit{Synopsis:} Pairs of (connected or disconnected) nodes may 
	show the tendency to share `followings'
\end{frame}

\begin{frame}{Cross-Level Effects: From Following to Taste and Back}
	{Membership closure}
	\centering
	\input{images/memberhip_closure.pgf}

	\raggedright
	\small 
	\textit{Synopsis:} $1$ introduces $0$ to music style $B$;
        $\{A, B\}$ is the set of music styles; $\{0, 1\}$ is
	the set of users; `vertical' ties can be intended as a bipartite 
	graph linking users and music styles via `likes,' `comments,' or
	`reposts' (mainly, user preferences); dashed arrows denote ties 
	at risk of emerging based on membership closure
\end{frame}

\begin{frame}{Cross-Level Effects: From Following to Preferences and Back}
	{Focal closure}
	\centering
	\input{images/focal_closure.pgf}

	\raggedright
	\small
	\textit{Synopsis:} $0$ follows $1$ because of shared taste/preferences; 
	$\{A, B\}$ is the set of music styles; $\{0, 1\}$ is
	the set of users; `vertical' ties can be intended as a bipartite 
	graph linking users and music styles via `likes,' `comments,' or
	`reposts'; dashed arrows denote ties at risk of emerging based on 
	focal closure 
\end{frame}

\subsection{Session 7}

\begin{frame}{Let Us Get Back to the Fundamental Question}{}
\centering
\Large 
Why do networks look how they look?
\end{frame}

\begin{frame}{A Possible Answer: Homophily}{}
	\begin{columns}
		\begin{column}{0.5\textwidth}
		\begin{itemize}
			\item 
			According to the homophily principle, similarity breeds 
			connection
			\pause 
			\item 
			Hence, networks are homogeneous concerning:
			\begin{itemize}
				\item 
				Sociodemographic variables (e.g., age)
			 	\item 
				Reference social categories (e.g., being a knowledge worker)
				\item 
				Behaviors (e.g., purchasing choices)
			\end{itemize}
		\end{itemize}
		\end{column}
		\begin{column}{0.5\textwidth}
			\includegraphics[width=1\textwidth]{images/birds_of_a_feather.jpeg}
		\end{column}
	\end{columns}
\end{frame}

\begin{frame}{Theoretical Background}{}
	\centering
	\includegraphics[width=0.5\textwidth]{images/mcpherson_et_al.png}

	\small
	Source is \cite{mcpherson2001}
\end{frame}

\begin{frame}{Homophily in Practice: Example 1}
	{Friendship and racial segregation in an American high-school}
	\centering 
	\includegraphics[width=0.75\textwidth]{images/moody_2001.png}

	\raggedright
	\small
	Source is \cite{moody_2001}; nodes are color-coded to reflect 
	student racial background; ties denote friendship between pairs 
	of nodes
\end{frame}

\begin{frame}{Homophily in Practice: Example 2}
	{Discovering cultural similarity is key for tie formation in speed dating}
	\begin{columns}[t]
		\begin{column}{0.5\textwidth}
			\centering
			\includegraphics[width=0.7\textwidth]{images/mcfarland_et_al.png}

			\footnotesize
			Source is \cite{mcfarland2013}
		\end{column}
		\begin{column}{0.5\textwidth}
			\centering
			\includegraphics[width=0.7\textwidth]{images/romantic_relation.jpg}
		\end{column}
	\end{columns}
\end{frame}

\begin{frame}{Homophily in Practice: Example 3}
	{Executives ask for advice from similar others in difficult times}
	\centering
	\includegraphics[width=0.8\textwidth]{images/mcdonald_westphal.png}

	\footnotesize
	Source is \cite{mcdonald2003}
\end{frame}

\begin{frame}{When Does a Network Exhibit Homophily?}{}
	\centering 
	\includegraphics[width=0.6\textwidth]{images/fig_4_2.png}

	\footnotesize
	\textit{Notes:} --- Nodes are color-coded concerning a key 
	relevant attribute (e.g., gender)
\end{frame}

%\begin{frame}{There Are Two Families of Homophily Measures}{}
%	\begin{columns}
%		\begin{column}{0.5\textwidth}
%			\centering 
%			\textbf{Network-level measures}
%
%			\includegraphics[width=0.7\textwidth]{images/newman_2003.png}
%
%			Source is ...
%
%		\end{column}
%		\begin{column}{0.5\textwidth}
%			\centering 
%			\textbf{Dyadic-level measures}
%			
%			\includegraphics[width=0.7\textwidth]{images/ertug_et_al_2018.png}
%
%			Source is ... 
%		\end{column}
%	\end{columns}
%\end{frame}

\begin{frame}{A Simple Homophily Measure Based on Frequencies}{Intuition}
	Let us consider `gender' as the key feature of nodes: what would 
	it mean for a network not to exhibit homophily?
	\begin{itemize}
		\item 
		The proportion of male and female friends that a person has 
		resembles the underlying male/female distribution in the full 
		population
		\pause 
		\item 
		In other words, ``if we were to randomly assign each node a 
		gender according to the gender balance in the real network, then 
		the number of cross-gender edges should not change 
		significantly relatively to what is seen in the real network''
	\end{itemize}

Source is 
\end{frame}

\begin{frame}{A Simple Homophily Measure Based on Frequencies}
	{The formal representation}
	Let's consider a network containing a fraction $p$ of male nodes ($m$) 
	and a fraction $q$ of female nodes ($f$), and denote the link between 
	nodes $i$ and $j$ as $e_{i,j}$.

	\vspace{2em}
	
	The no-homophily hypothesis implies what follows:
	\begin{enumerate}
	\item $pr(e_{ij} \ | \ i = m, \ j = m) = p * p$
	\item $pr(e_{ij} \ | \ i = f, \ j = f) = q * q$
	\item $pr(e_{ij} \ | \ i = m, \ j = f) = p * q$
	\item $pr(e_{ij} \ | \ i = f, \ j = m) = q * p$
	\end{enumerate}

	\vspace{2em}

	If the proportion of male-female ties deviates from $2*(p*q)$, then there is evidence of homophily in the network.

\end{frame}

\begin{frame}{Mechanisms Underlying the Homophily Principle}{}
	There are two mechanisms through which homophily affects tie formation:
	
	\begin{itemize}
	\item \textbf{Selection:} individual bond with others they perceive `similar' on 
	salient dimensions
	\item \textbf{Socialization:} individuals that directly, closely, frequently 
	interact with each other become more and more `similar'
	\end{itemize}

	\vspace{1em}

\small \textit{Note:} The relative strength of `selection' and `socialization' largely 
depends on the attributes individuals use to assess interpersonal similarity. 
Socialization does not operate for fixed demographic factors, whereas it 
plays a central role when attitudes or behavior are used as a basis for similarity 
\end{frame}

\begin{frame}{Mechanisms Underlying the Homophily Principle}{The socialization mechanism}
	\centering 

	\small Probability to become a member of a community $i$ when $k$ friends 
	are already part

	\includegraphics[width=0.65\textwidth]{images/membership_closure.png}

	\footnotesize Source is \cite{kossinets2009}
\end{frame}

\begin{frame}{Mechanisms Underlying the Homophily Principle}{The selection mechanism}
	\centering 

	\small Probability of inter-personal link formation when two members 
	share $k$ classes

	\includegraphics[width=0.7\textwidth]{images/focal_closure.png}

	\footnotesize Source is \cite{kossinets2009}
\end{frame}

\begin{frame}{Mechanisms Underlying the Homophily Principle}{The interaction 
	between selection and socialization}
	\centering 

	\small Similarity of two Wikipedia contributors' edits before and after 
	the formation of an inter-personal link

	\includegraphics[width=0.5\textwidth]{images/selection_and_socialization.png}

	\footnotesize Source is \cite{easley2010}
\end{frame}

% ================== Modeling selection and social influence ================
\section{Modeling Selection and Social Influence with SAOMs}

\subsection{The Nuts and Bolts of SAOMs}

\begin{frame}{Stochastic Actor Oriented Models}{The `what' and `who' aspects}
	\begin{itemize}
		\item 
		\textbf{Scope:} SAOMs are a family of models that 
		express empirically observed changes in network ties and 
		inmany cases, changes in individual attributes as 
		time-aggregated outcomes of a series of individual decisions 
		\cite{snijders_2016}
		\item
		\textbf{The role of actors:} SAOMs model network change over time from the 
		perspective of each actor, under the assumption that each actor 
		has the option to adjust the structure of his or her network 
		(by ``deciding'' whom to start, continue, or stop communicating 
		with) or the level of a specific attribute (whether to increase 
		or decrease the extent to which he or she perceives the team's 
		climate as being positive). 
	\end{itemize}
\end{frame}

\begin{frame}{Stochastic Actor Oriented Models}{The `how' aspects}
\begin{itemize}
	\item SAOMS enable researchers to use data from a first measured 
	timepoint to test whether a given set of hypothesized effects can 
	produce the network structures and attribute levels measured at 
	later timepoints
	\item Central to these models is the idea that changes in network ties 
	and actors' attributes occur continuously even though data on the state 
	of the network and its actors are collected at discrete timepoints
	\item These models also assume that the difference between observed 
	timepoints can be broken down into probabilistic, sequential small 
	steps, called ministeps
	\begin{itemize}
		\item At each ministep, a focal actor is randomly 
	         selected and has the opportunity to make a single decision
		\item In a network ministep, the actor has the opportunity to 
		modify one of his or her outgoing ties by creating a tie to a 
		new actor, terminating an existing tie, or maintaining current ties
		\item In an attribute (or behavioral) ministep, the actor can 
		modify (increase, decrease, or maintain) his or her level of a 
		given attribute
	\end{itemize}
\end{itemize}
\end{frame}

\begin{frame}{Modeling Network and Behavioral Change with Rules}{}
	\begin{itemize}
	\item To model these changes, the researcher constructs, on the basis of 
	theoretical and empirical considerations, a set of ``rules'' that 
	might drive an actor's decision to change a network tie or adjust
	the level of one of his or her attributes
	\item Rules are grouped into four categories:
		\begin{itemize}
		\item Network evolution rules 
		\item Attribute evolution rules 
		\item Social selection rules 
		\item Social influence rules 
		\end{itemize}
	\end{itemize}
\end{frame}

\begin{frame}{Modeling Network and Behavioral Change with Rules}{Network evolution rules}
	\begin{itemize}
	\item These rules determine \textbf{how ties develop given the structure of ties in 
	the previous timepoints}
	\item An example might be that individuals prefer 
	to communicate with those who communicated with them in previous 
	timepoints or that individuals prefer to communicate with those with 
	whom many others communicated in previous timepoints
	\end{itemize}
\end{frame}


\begin{frame}{Modeling Network and Behavioral Change with Rules}{Attribute evolution rules}
	\begin{itemize}
		\item  These rules determine how attributes evolve over time
		\item An example might be that there is a general tendency for 
		the attribute (negative affectivity, turnover intention, 
		identification with a program) to increase over time
	\end{itemize}
\end{frame}

\begin{frame}{Modeling Network and Behavioral Change with Rules}{Social selection rules}
	\begin{itemize}
		\item These rules describe how ties develop in response to 
		actors' attributes (and ties) in the previous timepoint
		\item For these rules, actors' attributes are considered to be 
		the drivers of network ties
		\item An example might be that people who have higher levels of 
		negative affectivity talk to fewer others over time or that
		people prefer to communicate with others whose levels of 
		turnover intentions resemble their own
	\end{itemize}
\end{frame}

\begin{frame}{Modeling Network and Behavioral Change with Rules}{Social influence rules}
\begin{itemize}
	\item These rules describe how actors' attributes change in response 
	to ties (and attributes) in the previous timepoint
	\item For these rules, the network is considered to drive actors' 
	attributes
	\item For example, communicating with fewer others over time may 
	increase a person's turn-over intentions, or people may catch their 
	network partners' levels of negative affectivity over time (emotional 
	contagion)
\end{itemize}
\end{frame}

\begin{frame}{Empirical Estimation Aspects}{Decision frequency and rules}
	\begin{itemize}
		\item SAOMs assume that whereas network and attribute data 
		are collected at discrete timepoints, underlying time is 
		continuous
		\item At discrete, unobservable timepoints, a randomly 
		selected actor gets an opportunity to change his or her outgoing 
		ties (a network ministep) or level of the attribute (a 
		behavioral ministep)
		\item This implies two separate subprocesses that need to be 
		modeled:
		\begin{itemize}
			\item The first subprocess involves modeling the 
			frequency of change: how often a ministep occurs
			\item The second subprocess models how change occurs: 
			what changes occur once an actor is given an opportunity 
			to change his or her network or attribute (in a network 
			or behavioral ministep, respectively)
		\end{itemize}
	\end{itemize}
\end{frame}

\begin{frame}{From Subprocesses to Objective Functions}
	The above mentiond subprocesses are modeled as two interdependent 
	mathematical functions termed objective functions:
	
	
	\begin{equation}
		f_{i}^{net}(x, y) = \sum_{k}\beta_{k}{net}s_{ik}^{net}
	\end{equation}

	\begin{equation}
		f_{i}^{beh}(x, y) = \sum_{k}\beta_{k}{beh}s_{ik}^{beh}
	\end{equation}

	where equation 1 denotes the objective function that actor $i$ seeks 
	to optimize in a network ministep, while equation 2 represents the 
	objective function that actor $i$ seeks to optimize in an attribute 
	(behavioral) ministep.

	\small Notes: i) in each equation, $f_{i}(x,z)$ denotes the value of the objective 
	function for actor $i$ for a given network state $x$ and $i$'s level of the 
	attribute $z$; ii) The value of $f$ is dependent on a series of parameter 
	values $b_{k}$, each of which is coupled to an effect, denoted $s_{ik}$; 
	iii) n effect represents a subgraph count in the network neighborhood of 
	the focal actor in a given ministep 
\end{frame}

\subsection{Modeling Friendship and Intention to Quit in the Workplace}

\begin{frame}{}{}
	\Large 
	\centering
	Let us model the co-evolution of a friendship network and intention 
	to quit.
\end{frame}


% ============================ Bibliography =================================
\begin{frame}
	\frametitle{References}
	\printbibliography
 \end{frame} 

% =========================== Document end ================================== 
\end{document}