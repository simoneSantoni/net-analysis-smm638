\documentclass[notes, aspectratio=1610]{beamer}
%\documentclass[aspectratio=1610]{beamer}

% ============================ Custom colors ================================
\definecolor{base_c}{rgb}{0.6,0,0}
\definecolor{comp_c}{rgb}{0.09803921568627451, 0.6901960784313725, 0.7529411764705882}
\definecolor{tri_1}{rgb}{0.09803921568627451, 0.7686274509803922, 0.19215686274509805}
\definecolor{tri_2}{rgb}{0.19215686274509805, 0.09803921568627451, 0.7686274509803922}


% ========================= Theme =========================================
\usetheme{Berkeley}
\usecolortheme{spruce}

% ========================= Essential packages ============================
%\usepackage{hyperref}
%\hypersetup{
%    colorlinks = true,
%    linkcolor = blue,
%    citecolor = blue,
%    filecolor = blue,
%    urlcolor = blue
%}

% ========================= Frame notes systm ============================
%\usepackage{pgfpages}
%\setbeameroption{show notes on second screen}

% ========================= Plotting ======================================
\usepackage{calc}
\usepackage{tikz}
\usetikzlibrary{arrows,
                arrows.meta,
                calc,
		chains,
                quotes,
                positioning,
		shapes,
                shapes.geometric}
\usepackage{graphicx}
\usepackage{graphics}
\usepackage{pgfplots}
\pgfplotsset{width=7cm,compat=1.17}
\usepackage{venndiagram}

%% ============================== Tabular =================================
\usepackage{booktabs}
\usepackage{tabularx,ragged2e}
\usepackage{array}
\usepackage{multirow}
\usepackage{siunitx}
  \sisetup{detect-all}
\usepackage{adjustbox}
\usepackage{rotating}
\usepackage{threeparttable}
\usepackage[justification=centering]{caption}

%% ========================== Coding snippets =============================
% Default fixed font does not support bold face
%\usepackage{minted}
%\usemintedstyle{vs}

% ========================= Infor on authors ==============================
\title[Network Theory and Network Data]
{The `What' and `How' of Network Analytics}
\subtitle{Network Theory and Network Data}
\author{S.~Santoni\inst{1}\inst{2}}
\institute{
	\inst{1}%
	Bayes Business School
	\and
	\inst{2}%
	Soundcloud
	}
\date{MSc in Business Analytics, 2022/23}

% ========================= TOC  ==========================================
\AtBeginSection[]
{
	\begin{frame}
		       \frametitle{Outline}
		       \tableofcontents[currentsection,currentsubsection]
	\end{frame}
}

% ========================= References ===================================
\usepackage[style=numeric,backend=biber]{biblatex}
\addbibresource{bibliography.bib}

% =========================== TOC =========================================
\AtBeginSubsection[]
{
    \begin{frame}
        \frametitle{Outline}
        \tableofcontents[currentsection,currentsubsection]
    \end{frame}
}

% ========================= Document  ====================================
\begin{document}

\begin{frame}
	\titlepage
\end{frame}

\begin{frame}{Outline}
	\tableofcontents
\end{frame}

% =========================== Session 1 wrapup =============================
\section{Session 1 Wrap Up}

\begin{frame}
% TODO: alternate colors in the table's rows
	\frametitle{There Are Several Families of Networks}
	\begin{table}
		\begin{small}
		\begin{center}
		\begin{tabular}[c]{|l|l|}
			\hline
			\textbf{Family} & 
			\textbf{Example} \\
			\hline 
			Biological networks\dotfill
			& A living organism's neural system\\
		        Cultural networks\dotfill
			& A model of the `returns of education'\\
			Financial networks\dotfill
			& A cryptocurrency\\
		        Information networks\dotfill
			& Information sharing among BA students\\
			Inter-organizational networks\dotfill
			& Technological alliances among pharma industry players\\
		        Organizational networks\dotfill
			& Knowledge sharing among financial analysists\\
		        Social networks\dotfill
			& Friendship among BA students\\
		        Transportation networks\dotfill
			& The Tube\\				
			\hline
		\end{tabular}
		\end{center}
		\end{small}
	\end{table}
\end{frame}

\begin{frame}[t]
% TODO: creat text box around `!! Pay Attention !!' 
	\frametitle{Networks Have a `Hard' and a `Soft' Component}
	\begin{small}
	\begin{columns}[t]
		\begin{column}{0.45\textwidth}
			\begin{center}
				\textbf{The hard component}
			\end{center}

			A network is a collection of nodes and edges, what is
			formally called a `graph':

			\vspace{-0.75em}

			\begin{equation}
				G = \{V, E\}
			\end{equation}

			where $V$ is the array of nodes 

			\vspace{1em}

			$\{v_{1}, v_{2}, ... , v_{i}, ... , v_{N}\}$
			
			\vspace{1em}
			
			and $E$ is the set of edges reflecting connections among
			pairs of nodes

			\vspace{1em}
			
			$\{..., \{v_{i}, v_{j}\}, \{v_{i}, v_{k}\}, ...\}$
		\end{column}
		\begin{column}{0.45\textwidth}
			\begin{center}
				\textbf{The soft component}
			\end{center}

			The soft component is the relationship that maps 
			the connections onto the pairs of nodes.
			Examples of relationships are affilition to a club,
		        music collab (i.e., a `feat'), friendship, marriage, 
			mentoring, tube route.

			\begin{center}
			!! Pay attention !!
			\end{center}

			\emph{A network is more than a graph}. 
			Two nodes may be connected for many reasons --- 
			when it come to analyze network data, we must be 
			specific about the concrete relationship under 
			investigation.
		\end{column}
	\end{columns}
\end{small}
\end{frame}

\begin{frame}
	\frametitle{A Real-World Example: The Soundcloud Networks}
	\begin{columns}
		\begin{column}{0.55\textwidth}
			\begin{figure}
				\begin{small}
					\begin{center}
						\includegraphics[
							width=0.95\textwidth
							]{example-image}
					\end{center}
				\end{small}
			\end{figure}
			
		\end{column}
		\begin{column}{0.4\textwidth}
			\begin{small}

			Some key general points emerging from the analysis of the 
			Soundcloud example:
			\begin{itemize}
				\item
				The same pair of nodes can be connected 
				because of multiple relatonships (i.e., `like,'
				`repost,' `comment')
				\item 
				The nodes of a network may have the 
				same type (e.g., `following') or 
				different types (e.g., `like')
				\item Analytically seprated networks may be 
				correlated (e.g., one tends to like her/his 
				followings' likes)
			\end{itemize}

			\end{small}
		\end{column}
	\end{columns}
\end{frame}

% =========================== Structuring a NA project =====================
\section{The Structure of Network Analytics Projects}

\begin{frame}
% This frame presents a Venn Diagram illustrating the three components of a
% 	Network Analytics Project:
%
%	1 - the business problem 
%	2 - network theory 
% 	3 - network 
% 
	\frametitle{What Are the Components of a Network Analytics' Project?}
	\begin{figure}
		% \begin{tikzpicture}
			\begin{center}
				\includegraphics[width=0.7\textwidth]{
					example-image
					}
			\end{center}
		% \end{tikzpicture}
	\end{figure}	
\end{frame}

% =========================== The business problem ========================
\section{The Business Problem}

\begin{frame}
% TODO: make indvidual cells colored
	\frametitle{}
	\begin{table}
		\begin{small}
			\begin{center}
				\begin{tabular}[c]{
					|l|p{2cm}|p{2cm}|p{2cm}|p{2cm}|
					}
					\hline
					\textbf{}
					& \multicolumn{4}{c|}{\textbf{Level}}\\
					\hline
					\textbf{Objective function}
					& \textbf{Individual}
					& \textbf{Team}
					& \textbf{Organization}
					& \textbf{Inter-orgs}\\
					\hline
					Coordination\dotfill
					& 
					&
					&
					&\\ 
					Knowledge sharing\dotfill
					&
					&
					&
					&\\
					Task performance\dotfill
					&
					&
					&
					&\\ 
					Innovation\dotfill
					&
					&
					&
					&\\ 
				        Economic performance\dotfill
					&
					&
					&
					&\\
					\hline
				\end{tabular}
			\end{center}
		\end{small}
	\end{table}
	
\end{frame}

\begin{frame}
	\frametitle{Sample of Real Business Problems Raised by Industry Partners}
\end{frame}

% =========================== Network theory ==============================
\section{Network Theory}

% =========================== Network data ================================
\section{Network Data}

% =========================== Session 2 wrapup =============================
\section{Session 2 Wrap Up}

% =========================== Bibliography =================================

\begin{frame}
	\frametitle{References}
	\printbibliography
 \end{frame} 

\end{document}